\part{Inference}

\chapter{Bayesian Data Analysis}\label{bayesian.chapter}

\noindent
\cite{GelmanEtAl:2013} provide the following
characterization of Bayesian data analysis.
%
\begin{quote}
  By Bayesian data analysis, we mean practical methods for making
  inferences from data using probability models for quantities we
  observe and about which we wish to learn.
\end{quote}
%
They go on to describe how Bayesian statistics differs from
frequentist approaches.
%
\begin{quote}
  The essential characteristic of Bayesian methods is their explicit
  use of probability for quantifying uncertainty in inferences based
  on statistical analysis.
\end{quote}
%
Because they view probability as the limit of relative frequencies of
observations, strict frequentists forbid probability statements about
parameters.  Parameters are considered fixed, not random.

Bayesians also treat parameters as fixed but unknown.  But unlike
frequentists, they make use of both prior distributions over
parameters and posterior distributions over parameters.  These prior
and posterior probabilities and posterior predictive probabilities are
intended to characterize knowledge about the parameters and future
observables.  Posterior distributions form the basis of Bayesian
inference, as described below.

\section{Bayesian Modeling}

\citep{GelmanEtAl:2013} break applied Bayesian modeling
into the following three steps.
%
\begin{enumerate}
\item  Set up a full probability model for all observable and
  unobservable quantities.  This model should be consistent with
  existing knowledge of the data being modeled and how it was
  collected.
\item Calculate the posterior probability of unknown quantities
  conditioned on observed quantities.  The unknowns may include
  unobservable quantities such as parameters and potentially
  observable quantities such as predictions for future observations.
\item Evaluate the model fit to the data.  This includes evaluating
  the implications of the posterior.
\end{enumerate}
%
Typically, this cycle will be repeated until a sufficient fit is
achieved in the third step.  Stan automates the calculations involved
in the second and third steps.

\section{Bayesian Inference}

\subsection{Basic Quantities}

The mechanics of Bayesian inference follow directly from Bayes's rule.
To fix notation, let $y$ represent observed quantities such as data
and let $\theta$ represent unknown quantities such as parameters and
future observations.  Both $y$ and $\theta$ will be modeled as random.
Let $x$ represent known, but unmodeled quantities such as constants,
hyperparameters, and predictors.

\subsection{Probability Functions}

The probability function $p(y,\theta)$ is the joint probability
function of the data $y$ and parameters $\theta$.  The constants and
predictors $x$ are implicitly understood as being part of the
conditioning.  The conditional probability function $p(y|\theta)$ of
the data $y$ given parameters $\theta$ and constants $x$ is called the
sampling probability function; it is also called the likelihood
function when viewed as a function of $\theta$ for fixed $y$ and $x$.

The probability function $p(\theta)$ over the parameters given the
constants $x$ is called the prior because it characterizes the probability
of the parameters before any data is observed.  The conditional
probability function $p(\theta|y)$ is called the posterior because
it characterizes the probability of parameters given observed data $y$
and constants $x$.

\subsection{Bayes's Rule}

The technical apparatus of Bayesian inference hinges on the following
chain of equations, known in various forms as Bayes's rule (where
again, the constants $x$ are implicit).
%
\[
\begin{array}{rcll}
p(\theta|y)  & =  & \displaystyle \frac{p(\theta,y)}{p(y)}
& \mbox{{} \ \ \ \ \ [definition of  conditional probability]}
\\[16pt]
& = & \displaystyle \frac{p(y|\theta) \, p(\theta)}{p(y)}
& \mbox{{} \ \ \ \ \ [chain rule]}
\\[16pt]
& = & \displaystyle \frac{p(y|\theta) \, p(\theta)}
                        {\int_{\Theta} p(y,\theta) \, d\theta}
& \mbox{{} \ \ \ \ \ [law of total probability]}
\\[16pt]
& = & \displaystyle \frac{p(y|\theta) \, p(\theta)}
                        {\int_{\Theta} p(y|\theta) \, p(\theta) \, d\theta}
& \mbox{{} \ \ \ \ \ [chain rule]}
\\[16pt]
& \propto & \displaystyle p(y|\theta) \, p(\theta)
& \mbox{{} \ \ \ \ \ [$y$ is fixed]}
\end{array}
\]
%
Bayes's rule ``inverts'' the probability of the posterior
$p(\theta|y)$, expressing it solely in terms of the likelihood
$p(y|\theta)$ and prior $p(\theta)$ (again, with constants and
predictors $x$ implicit).  The last step is important for Stan, which
only requires probability functions to be characterized up to a
constant multiplier.

\subsection{Predictive Inference}

The uncertainty in the estimation of parameters $\theta$ from the data
$y$ (given the model) is characterized by the posterior $p(\theta|y)$.
The posterior is thus crucial for Bayesian predictive inference.

If $\tilde{y}$ is taken to represent new, perhaps as yet unknown,
observations, along with corresponding constants and predictors
$\tilde{x}$, then the posterior predictive probability function is
given by
%
\[
p(\tilde{y}|y)
= \int_{\Theta} p(\tilde{y}|\theta)
                \, p(\theta|y) \, d\theta.
\]
Here, both the original constants and predictors $x$ and the new
constants and predictors $\tilde{x}$ are implicit.  Like the posterior
itself, predictive inference is characterized probabilistically.
Rather than using a point estimate of the parameters $\theta$,
predictions are made based on averaging the predictions over a range
of $\theta$ weighted by the posterior probability $p(\theta|y)$ of
$\theta$ given data $y$ (and constants $x$).

The posterior may also be used to estimate event probabilities.  For
instance, the probability that a parameter $\theta_k$ is greater than
zero is characterized probabilistically by
%
\[
\mbox{Pr}[\theta_k > 0]
= \int_{\Theta} \mbox{I}(\theta_k > 0) \, p(\theta|y) \, d\theta.
\]
%
The indicator function, $\mbox{I}(\phi)$, evaluates to one if the
proposition $\phi$ is true and evaluates to zero otherwise.

Comparisons involving future observables may be carried out in
the same way.  For example, the probability that $\tilde{y}_n >
\tilde{y}_{n'}$ can be characterized using the posterior predictive
probability function as
\[
\mbox{Pr}[\tilde{y}_n > \tilde{y}_{n'}]
= \int_{\Theta} \int_{Y} \mbox{I}(\tilde{y}_n > \tilde{y}_{n'}) \,
p(\tilde{y}|\theta) p(\theta|y) \, d\tilde{y} \, d\theta.
\]


\subsection{Posterior Predictive Checking}

After the parameters are fit to data, they can be used to simulate a
new data set by running the model inferences in the forward
direction.  These replicated data sets can then be compared to the
original data either visually or statistically to assess model fit
\citep[Chapter 6]{GelmanEtAl:2013}.

In Stan, posterior simulations can be generated in two ways.  The
first approach is to treat the predicted variables as parameters and
then define their distributions in the model block.  The second
approach, which also works for discrete variables, is to generate
replicated data using random-number generators in the generated
quantities block.



\chapter{Penalized Maximum Likelihood Point Estimation}\label{mle.chapter}

\noindent
This chapter defines the workhorses of non-Bayesian estimation,
maximum likelihood and penalized maximum likelihood, and relates them
to Bayesian point estimation based on posterior means, medians, and
modes.  Such estimates are called ``point estimates'' because they
are composed of a single value for the model parameters $\theta$
rather than a posterior distribution.

Stan's optimizer can be used to implement (penalized) maximum
likelihood estimation for any likelihood function and penalty function
that can be coded in Stan's modeling language.  Stan's optimizer can
also be used for point estimation in Bayesian settings based on
posterior modes.  Stan's Markov chain Monte Carlo samplers can be used
to implement point inference in Bayesian models based on posterior
means or medians.

\section{Maximum Likelihood Estimation}\label{mle.section}

Given a likelihood function $p(y|\theta)$ and a fixed data vector $y$,
the maximum likelihood estimate (MLE) is the parameter vector $\hat{\theta}$
that maximizes the likelihood, i.e.,
\[
\hat{\theta} = \mbox{argmax}_{\theta} \ p(y|\theta).
\]
It is usually more convenient to work on the log scale.
An equivalent%
%
\footnote{The equivalence follows from the fact that densities are
  positive and the log function is strictly monotonic, i.e.,
  $p(y|\theta) \geq 0$ and for all $a, b > 0$, $\log a > \log b$ if and
  only if $a > b$.}
%
formulation of the MLE is
%
\[
\hat{\theta} = \mbox{argmax}_{\theta} \ \log p(y|\theta).
\]

\subsection{Existence of Maximum Likelihood Estimates}

Because not all functions have unique maximum values, maximum
likelihood estimates are not guaranteed to exist.  As discussed in
\refchapter{problematic-posteriors}, this situation can arise when
%
\begin{itemize}
\item there is more than one point that maximizes the likelihood function,
\item the likelihood function is unbounded, or
\item the likelihood function is bounded by an asymptote that is never
  reached for legal parameter values.
\end{itemize}
%
These problems persist with the penalized maximum likelihood estimates
discussed in the next section, and Bayesian posterior modes as
discussed in the following section.


\subsection{Example: Linear Regression}

Consider an ordinary linear regression problem with an $N$-dimensional
vector of observations $y$, an $(N \times K)$-dimensional data matrix
$x$ of predictors, a $K$-dimensional parameter vector $\beta$ of
regression coefficients, and a real-valued noise scale $\sigma > 0$,
with log likelihood function
\[
\log p(y|\beta,x) = \sum_{n=1}^N \log \distro{Normal}(y_n|x_n \beta,
\sigma).
\]
%
The maximum likelihood estimate for $\theta = (\beta,\sigma)$ is just
\[
(\hat{\beta},\hat{\sigma})
\ = \
\mbox{argmax}_{\beta,\sigma}
\log p(y|\beta,\sigma,x) = \sum_{n=1}^N \log \distro{Normal}(y_n|x_n \beta, \sigma).
\]

\subsubsection{Squared Error}

A little algebra on the log likelihood function shows that the
marginal maximum likelihood estimate $\hat{\theta} =
(\hat{\beta},\hat{\sigma})$ can be equivalently formulated for
$\hat{\beta}$ in terms of least squares.  That is, $\hat{\beta}$ is
the value for the coefficient vector that minimizes the sum of squared
prediction errors,
%
\[
\hat{\beta}
\ = \
\mbox{argmin}_{\beta} \sum_{n=1}^N (y_n - x_n \beta)^2
\ = \
\mbox{argmin}_{\beta} (y - x \beta)^{\top} (y - x\beta).
\]
%
The residual error for data item $n$ is the difference between the
actual value and predicted value, $y_n - x_n \hat{\beta}$.  The
maximum likelihood estimate for the noise scale, $\hat{\sigma}$ is
just the square root of the average squared residual,
\[
\hat{\sigma}^2
\ = \
\frac{1}{N} \sum_{n=1}^N \left( y_n - x_n \hat{\beta} \right)^2
\ = \
\frac{1}{N} (y - x \hat{\beta})^{\top} (y - x\hat{\beta}).
\]

\subsubsection{Minimizing Squared Error in Stan}

The squared error approach to linear regression can be directly coded
in Stan with the following model.
%
\begin{stancode}
data {
  int<lower=0> N;
  int<lower=1> K;
  vector[N] y;
  matrix[N,K] x;
}
parameters {
  vector[K] beta;
}
transformed parameters {
  real<lower=0> squared_error;
  squared_error = dot_self(y - x * beta);
}
model {
  target += -squared_error;
}
generated quantities {
  real<lower=0> sigma_squared;
  sigma_squared = squared_error / N;
}
\end{stancode}
%
Running Stan's optimizer on this model produces the MLE for the linear
regression by directly minimizing the sum of squared errors and using
that to define the noise scale as a generated quantity.

By replacing \code{N} with \code{N-1} in the denominator of the
definition of \code{sigma\_squared}, the more commonly supplied
unbiased estimate of $\sigma^2$ can be calculated; see
\refsection{estimation-bias} for a definition of estimation bias and a
discussion of estimating variance.



\section{Penalized Maximum Likelihood Estimation}

There is nothing special about a likelihood function as far as the
ability to perform optimization is concerned.  It is common among
non-Bayesian statisticians to add so-called ``penalty'' functions
to log likelihoods and optimize the new function.  The penalized
maximum likelihood estimator for a log likelihood function
$\log p(y|\theta)$ and penalty function $r(\theta)$ is defined to be
\[
\hat{\theta} = \mbox{argmax}_{\theta} \log p(y|\theta) - r(\theta).
\]
The penalty function $r(\theta)$ is negated in the maximization so
that the estimate $\hat{\theta}$ balances maximizing the log
likelihood and minimizing the penalty.  Penalization is sometimes
called ``regularization.''


\subsection{Examples}\label{penalized-mle-examples}

\subsubsection{Ridge Regression}

Ridge regression \citep{HoerlKennard:1970} is based on penalizing the
Euclidean length of the coefficient vector $\beta$. The ridge penalty
function is
%
\[
r(\beta)
\ = \
\lambda \, \sum_{k=1}^K \beta_k^2
\ = \
\lambda \, \beta^{\top} \beta,
\]
%
where $\lambda$ is a constant tuning parameter that determines the
magnitude of the penalty.


Therefore, the penalized maximum likelihood estimate for ridge
regression is just
%
\[
(\hat{\beta},\hat{\sigma})
\ = \
\mbox{argmax}_{\beta,\sigma} \,
 \sum_{n=1}^N \log \distro{Normal}(y_n|x_n \beta, \sigma) - \lambda
 \sum_{k=1}^K \beta_k^2
\]
%
The ridge penalty is sometimes called L2 regularization or shrinkage,
because of its relation to the L2 norm.

Like the basic MLE for linear regression, the ridge regression
estimate for the coefficients $\beta$ can also be formulated in terms
of least squares,
%
\[
\hat{\beta}
\ = \
\mbox{argmin}_{\beta} \, \sum_{n=1}^N (y_n - x_n \beta)^2 + \sum_{k=1}^K \beta_k^2
\ = \
\mbox{argmin}_{\beta} \, (y - x\beta)^{\top} (y - x\beta) +
\lambda \beta^{\top} \beta.
\]

The effect of adding the ridge penalty function is that the ridge
regression estimate for $\beta$ is a vector of shorter length, or in
other words, $\hat{\beta}$ is shrunk.  The ridge estimate does not
necessarily have a smaller absolute $\beta_k$ for each $k$, nor does
the coefficient vector necessarily point in the same direction as the
maximum likelihood estimate.

In Stan, adding the ridge penalty involves adding its magnitude as a
data variable and the penalty itself to the model block,
%
\begin{stancode}
data {
  // ...
  real<lower=0> lambda;
}
// ...
model {
  // ...
  target += - lambda * dot_self(beta);
}
\end{stancode}
%
The noise term calculation remains the same.

\subsubsection{The Lasso}

The lasso \citep{Tibshirani:1996} is an alternative to ridge
regression that applies a penalty based on the sum of the absolute
coefficients, rather than the sum of their squares,
\[
r(\beta) = \lambda \sum_{k=1}^K | \beta_k |.
\]
The lasso is also called L1 shrinkage due to its relation to the L1
norm, which is also known as taxicab distance or Manhattan distance.

Because the derivative of the penalty does not depend on the value of
the $\beta_k$,
\[
\frac{d}{d\beta_k} \lambda \sum_{k=1}^K | \beta_k | =
\mbox{signum}(\beta_k),
\]
it has the effect of shrinking parameters all the way to 0 in maximum
likelihood estimates.  Thus it can be used for variable selection as
well as just shrinkage.%
%
\footnote{In practice, Stan's gradient-based optimizers are not
  guaranteed to produce exact zero values; see
  \cite{LangfordEtAl:2009} for a discussion of getting exactly zero
  values with gradient descent.}
%
The lasso can be implemented in Stan just as easily as ridge
regression, with the magnitude declared as data and the penalty added
to the model block,
%
\begin{stancode}
data {
  // ...
  real<lower=0> lambda;
}
// ...
model {
  // ...
  for (k in 1:K)
    target += - lambda * fabs(beta[k]);
}
\end{stancode}

\subsubsection{The Elastic Net}

The naive elastic net \citep{ZouHastie:2005} involves a weighted
average of ridge and lasso penalties, with a penalty function
\[
r(\beta)
= \lambda_1 \sum_{k=1}^K |\beta_k|
+ \lambda_2 \sum_{k=1}^K \beta_k^2.
\]
The naive elastic net combines properties of both ridge regression and
the lasso, providing both identification and variable selection.

The naive elastic net can be implemented directly in Stan by combining
implementations of ridge regression and the lasso, as
%
\begin{stancode}
data {
  real<lower=0> lambda1;
  real<lower=0> lambda2;
  // ...
}
// ...
model {
  // ...
  for (k in 1:K)
    target += -lambda1 * fabs(beta[k]);
  target += -lambda2 * dot_self(beta);
}
\end{stancode}
%
Note that the signs are negative in the program because $r(\beta)$ is
a penalty function.

The elastic net \citep{ZouHastie:2005} involves adjusting the final estimate for
$\beta$ based on the fit $\hat{\beta}$ produced by the naive elastic
net.  The elastic net estimate is
\[
\hat{\beta} = (1 + \lambda_2) \beta^*
\]
where $\beta^{*}$ is the naive elastic net estimate.

To implement the elastic net in Stan, the data, parameter, and model
blocks are the same as for the naive elastic net.  In addition, the
elastic net estimate is calculated in the generated quantities block.
%
\begin{stancode}
generated quantities {
  vector[K] beta_elastic_net;
  // ...
  beta_elastic_net = (1 + lambda2) * beta;
}
\end{stancode}
%
The error scale also needs to be calculated in the generated
quantities block based on the elastic net coefficients
\code{beta\_elastic\_net}.


\subsubsection{Other Penalized Regressions}

It is also common to use penalty functions that bias the coefficient
estimates toward values other than 0, as in the estimators of
\cite{JamesStein:1961}.  Penalty functions can also be used to bias
estimates toward population means; see
\citep{EfronMorris:1975,Efron:2012}.  This latter approach is similar
to the hierarchical models commonly employed in Bayesian statistics.


\section{Estimation Error, Bias, and Variance}\label{estimation-bias.section}

An estimate $\hat{\theta}$ depends on the particular data $y$ and
either the log likelihood function, $\log p(y|\theta)$, penalized log
likelihood function $\log p(y|\theta) - r(\theta)$, or log probability
function $\log p(y,\theta) = \log p(y,\theta) + \log p(\theta)$.  In
this section, the notation $\hat{\theta}$ is overloaded to indicate
the estimator, which is an implicit function of the data and
(penalized) likelihood or probability function.

\subsection{Estimation Error}

For a particular observed data set $y$ generated according to true
parameters $\theta$, the estimation error is the difference between
the estimated value and true value of the parameter,
\[
\mbox{err}(\hat{\theta}) = \hat{\theta} - \theta.
\]
%

\subsection{Estimation Bias}

For a particular true parameter value $\theta$ and a likelihood
function $p(y|\theta)$, the expected estimation error averaged over
possible data sets $y$ according to their density under the likelihood
is
%
\[
\mathbb{E}_{p(y|\theta)}[\hat{\theta}]
\ = \
\int \left( \mbox{argmax}_{\theta'} p(y|\theta') \right) p(y|\theta) dy.
\]

An estimator's bias is the expected estimation error,
%
\[
\mathbb{E}_{p(y|\theta)}[\hat{\theta} - \theta]
\ = \
\mathbb{E}_{p(y|\theta)}[\hat{\theta}] - \theta
\]
%
The bias is a multivariate quantity with the same dimensions as
$\theta$.  An estimator is unbiased if its expected estimation error
is zero and biased otherwise.

\subsubsection{Example: Estimating a Normal Distribution}

Suppose a data set of observations $y_n$ for $n \in 1{:}N$ drawn from
a normal distribution.  This presupposes a model $y_n \sim
\distro{Normal}(\mu,\sigma)$, where both $\mu$ and $\sigma > 0$ are
parameters.  The log likelihood is just
\[
\log p(y|\mu,\sigma) = \sum_{n=1}^N \log
\distro{Normal}(y_n|\mu,\sigma).
\]
The maximum likelihood estimator for $\mu$ is just the sample mean,
i.e., the average of the samples,
\[
\hat{\mu} = \frac{1}{N} \sum_{n=1}^N y_n.
\]
The maximum likelihood estimate for the mean is unbiased.

The maximum likelihood estimator for the variance $\sigma^2$ is the
average of the squared difference from the mean,
\[
\hat{\sigma}^2 = \frac{1}{N} \sum_{n=1}^N (y_n - \hat{\mu})^2.
\]
The maximum likelihood for the variance is biased on the low side,
i.e.,
%
\[
\mathbb{E}_{p(y|\mu,\sigma)}[\hat{\sigma}^2] < \sigma.
\]
%
The reason for this bias is that the maximum likelihood estimate is
based on the difference from the estimated mean $\hat{\mu}$.  Plugging
in the actual mean can lead to larger sum of squared differences;  if
$\mu \neq \hat{\mu}$, then
\[
\frac{1}{N} \sum_{n=1}^N (y_n - \mu)^2
>
\frac{1}{N} \sum_{n=1}^N (y_n - \hat{\mu})^2.
\]

An alternative estimate for the variance is the sample variance, which
is defined by
\[
\hat{\mu} = \frac{1}{N-1} \sum_{n=1}^N (y_n - \hat{\mu})^2.
\]
This value is larger than the maximum likelihood estimate by a factor
of $N/(N-1)$.


\subsection{Estimation Variance}

The variance of component $k$ of an estimator $\hat{\theta}$ is
computed like any other variance, as the expected squared difference
from its expectation,
%
\[
\mbox{var}_{p(y|\theta})[\hat{\theta}_k]
\ = \
\mathbb{E}_{p(y|\theta})[\, (\hat{\theta}_k -
\mathbb{E}_{p(y|\theta)}[\hat{\theta}_k])^2 \,].
\]
%
The full $K \times K$ covariance matrix for the estimator is thus
defined, as usual, by
%
\[
\mbox{covar}_{p(y|\theta)}[\hat{\theta}]
\ = \
\mathbb{E}_{p(y|\theta})[\, (\hat{\theta} - \mathbb{E}[\hat{\theta}]) \,
                         (\hat{\theta} -
                         \mathbb{E}[\hat{\theta}])^{\top} \, ].
\]

Continuing the example of estimating the mean and variance of a normal
distribution based on sample data, the maximum likelihood estimator
(i.e., the sample mean) is the unbiased estimator for the mean $\mu$
with the lowest variance; the Gauss-Markov theorem establishes this
result in some generality for least-squares estimation, or
equivalently, maximum likelihood estimation under an assumption of
normal noise; see \citep[Section~3.2.2]{HastieTibshiraniFriedman:2009}.

\chapter{Bayesian Point Estimation}

There are three common approaches to Bayesian point estimation based
on the posterior $p(\theta|y)$ of parameters $\theta$ given observed
data $y$: the mode (maximum), the mean, and the median.

\section{Posterior Mode Estimation}

This section covers estimates based on the parameters $\theta$ that
maximize the posterior density, and the next sections continue with
discussions of the mean and median.

An estimate based on a model's posterior mode can be defined by
%
\[
\hat{\theta} = \mbox{argmax}_{\theta} \, p(\theta|y).
\]
%
When it exists, $\hat{\theta}$ maximizes the posterior density of the
parameters given the data.  The posterior mode is sometimes called the
``maximum a posteriori'' (MAP) estimate.

As discussed in \refchapter{problematic-posteriors} and
\refsection{mle}, a unique posterior mode might not
exist---there may be no value that maximizes the posterior mode or
there may be more than one.  In these cases, the posterior mode
estimate is undefined.  Stan's optimizer, like most optimizers, will
have problems in these situations.  It may also return a locally
maximal value that is not the global maximum.

In cases where there is a posterior mode, it will correspond to a
penalized maximum likelihood estimate with a penalty function equal to
the negation of the log prior.  This is because Bayes's rule,
\[
p(\theta|y) = \frac{p(y|\theta) \, p(\theta)}{p(y)},
\]
ensures that
%
\begin{eqnarray*}
\mbox{argmax}_{\theta} \ p(\theta|y)
& = &
\mbox{argmax}_{\theta} \ \frac{p(y|\theta) \, p(\theta)}{p(y)}
\\[6pt]
& = &
\mbox{argmax}_{\theta} \ p(y|\theta) \, p(\theta),
\end{eqnarray*}
%
and the positiveness of densities and the strict monotonicity of log
ensure that
\[
\mbox{argmax}_{\theta} \ p(y|\theta) \, p(\theta)
\ = \
\mbox{argmax}_{\theta} \ \log p(y|\theta) + \log p(\theta).
\]
%

In the case where the prior (proper or improper) is uniform, the
posterior mode is equivalent to the maximum likelihood estimate.

For most commonly used penalty functions, there are probabilistic
equivalents.  For example, the ridge penalty function corresponds to a
normal prior on coefficients and the lasso to a Laplace prior.  The
reverse is always true---a negative prior can always be treated as a
penalty function.



\section{Posterior Mean Estimation}

A standard Bayesian approach to point estimation is to use the
posterior mean (assuming it exists), defined by
%
\[
\hat{\theta} = \int \theta \, p(\theta|y) \, d\theta.
\]
%
The posterior mean is often called {\it the}\ Bayesian estimator,
because it's the estimator that minimizes the expected square error of
the estimate.

An estimate of the posterior mean for each parameter is returned by
Stan's interfaces;  see the RStan, CmdStan, and PyStan user's guides
for details on the interfaces and data formats.

Posterior means exist in many situations where posterior
modes do not exist.  For example, in the $\distro{Beta}(0.1, 0.1)$
case, there is no posterior mode, but posterior mean is well defined
with value 0.5.

A situation where posterior means fail to exist but posterior modes do
exist is with a posterior with a Cauchy distribution
$\distro{Cauchy}(\mu,\tau)$.  The posterior mode is $\mu$, but the
integral expressing the posterior mean diverges.  Such diffuse priors
rarely arise in practical modeling applications; even with a Cauchy
Cauchy prior for some parameters, data will provide enough constraints
that the posterior is better behaved and means exist.

Sometimes when posterior means exist, they are not meaningful, as in
the case of a multimodal posterior arising from a mixture model or in
the case of a uniform distribution on a closed interval.


\section{Posterior Median Estimation}

The posterior median (i.e., 50th percentile or 0.5 quantile) is
another popular point estimate reported for Bayesian models.  The
posterior median minimizes the expected absolute error of estimates.
These estimates are returned in the various Stan interfaces;  see the
RStan, PyStan and CmdStan user's guides for more information on
format.

Although posterior medians may fail to be meaningful, they often exist
even where posterior means do not, as in the Cauchy distribution.



\chapter{Variational Inference}\label{vi-advanced.chapter}

\noindent
Stan implements an automatic variational inference algorithm based on
transforming variables to the unconstrained scale and using a normal
approximating distribution.

Classical variational inference algorithms are difficult to derive. We
must first define the family of approximating densities, and then
calculate model-specific quantities relative to that family to solve
the variational optimization problem.  Both steps require expert
knowledge.  The resulting algorithm is tied to both the model and the
chosen approximation.

We begin by briefly describing the classical variational inference framework.
For a thorough exposition, please refer to
\citet{Jordan:1999,Wainwright-Jordan:2008}; for a textbook presentation, please
see \citet{Bishop:2006}. We follow with a high-level description of Automatic
Differentiation Variational Inference (ADVI). For more details, see
\citep{Kucukelbir:2015}.


\section{Classical Variational Inference}

Variational inference approximates the
posterior $p(\theta \, | \, y)$ with a simple, parameterized distribution
$q(\theta \, | \, \phi)$. It matches the approximation to the
true posterior by minimizing the Kullback-Leibler (KL) divergence,
%
\[
  \phi^* = \argmin_\phi
  \KL{q(\theta \, | \, \phi) }{ p(\theta \mid y)}.
\]
%
Typically the KL divergence lacks an analytic, closed-form solution.
Instead we maximize a proxy to the KL divergence, the evidence lower bound
(ELBO)
%
\[
  \mathcal{L} (\phi)
  =
  \E_{q (\theta)} \big[ \log p (y,\theta) \big]
  -
  \E_{q (\theta)} \big[ \log q (\theta\, | \,\phi) \big].
\]
%
The first term is an expectation of the log
joint density under the approximation, and the second is the entropy of the
variational density. Maximizing the ELBO minimizes the KL
divergence \citep{Jordan:1999,Bishop:2006}.


\section{Automatic Variational Inference}

ADVI maximizes the ELBO in the real-coordinate space. Stan transforms the
parameters from (potentially) constrained domains to
the real-coordinate space. We denote the combined transformation as
$T:\theta \to \zeta$, with the $\zeta$ variables living in $\mathbb{R}^K$.
The variational objective (ELBO) becomes
%
\[
  \mathcal{L}(\phi)
  =
  \E_{q(\zeta\,|\,\phi)}
  \bigg[
  \log p (y, T^{-1}(\zeta))
  +
  \log \big| \det J_{T^{-1}}(\zeta) \big|
  \bigg]
  -
  \E_{q (\zeta\, | \,\phi)} \big[ \log q (\zeta\, | \,\phi) \big].
\]
%
Since the $\zeta$ variables live in the real-coordinate space, we can choose a
fixed family for the variational distribution. We choose a fully-factorized
Gaussian,
%
\[
  q(\zeta \, | \, \phi)
  =
  \distro{Normal}\left(\zeta \, | \, \mu, \sigma\right)
  =
  \prod_{k=1}^K
  \distro{Normal}
  \left(\zeta_k \, | \, \mu_k, \sigma_k\right),
\]
%
where the vector
$\phi = (\mu_{1},\cdots,\mu_{K}, \sigma_ {1},\cdots,\sigma_{K})$
concatenates the mean and standard deviation of each Gaussian factor.
This reflects the ``mean-field'' assumption in classical variational
inference algorithms; we will refer to this particular decomposition
as the \texttt{meanfield} option.

The transformation $T$ maps the support of the parameters to the real
coordinate space. Thus, its inverse $T^{-1}$ maps back to the support of the
latent variables. This implicitly defines the variational approximation in the
original latent variable space as
%
\[
\distro{Normal} \left(T(\theta) \, | \, \mu, \sigma\right)
\big| \det J_{T}(\theta) \big|.
\]
This is, in general, not a Gaussian distribution.
This choice may call to mind the Laplace approximation
technique, where a second-order Taylor expansion around the
maximum-a-posteriori estimate gives a Gaussian approximation to the
posterior. However, they are not the same \citep{Kucukelbir:2015}.

The variational objective (ELBO) that we maximize is,
\[
  \mathcal{L}(\phi)
  =
  \E_{q(\zeta\, | \,\phi)}
  \bigg[
  \log p (y, T^{-1}(\zeta))
  +
  \log \big| \det J_{T^{-1}}(\zeta) \big|
  \bigg]
  +
  \sum_{k=1}^K \log \sigma_k,
\]
where we plug in the analytic form for the Gaussian entropy and drop all terms
that do not depend on $\phi$. We discuss how we perform the maximization in
\refchapter{vi-algorithms}.
